\prefacesection{Abstract}
\paragraph{}This Master's thesis presents results from two clinical hearing aid studies.  Wide dynamic range compression (WDRC), a hearing aid amplification algorithm widely used in hearing aid industry, is compared against a novel hearing aid called the Neuro-Compensator (NC), which employs a neural-based amplification algorithm based on a computational model of the auditory periphery.  The NC strategy involves preprocessing an incoming auditory signal, such that when the signal is presented to a damaged cochlea, auditory nerve output is reconstructed to look similar to the auditory nerve output of a healthy cochlea for the original auditory signal.

The NC and WDRC hearing aid technologies are compared across a multitude of auditory domains.  Objective measures of speech intelligibility in quiet and in noise, music perception, sound localization, and subjective measures of sound quality are obtained.

It was hypothesized that the NC would restore more normal auditory abilities across auditory domains, due to its proposed strategy of restoring more normal auditory nerve output.  Results from the clinical hearing aid studies quantified domains in which the NC was superior to WDRC, and vice versa. 